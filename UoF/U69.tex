{\it Tenta IT, 2015 april, uppg. 5}\\
Antag att $G=(\vec{g_1},\ \vec{g_2},\ \vec{g_3})$ är en bas där $||\vec{g_1}||=1,\ ||\vec{g_2}||=\sqrt{2},\ ||\vec{g_3}||=2$ och vinkeln $\angle\vec{g_1}\vec{g_2}$ är $\pi/4$, vinkeln $\angle\vec{g_1}\vec{g_3}$ är $\pi/3$ och vinkeln $\angle\vec{g_2}\vec{g_3}$ är $\pi/2$. Låt $\vec{u}$ och $\vec{v}$ vara de vektorer som i basen $G$ har koordinater
$$\vec{u_G}=\begin{bmatrix}1\\1\\2\end{bmatrix},\ \vec{v_G}=\begin{bmatrix}-1\\3\\1\end{bmatrix}$$
Beräkna vinkeln $\angle\vec{u}\vec{v}$ mellan dessa vektorer $\vec{u}$ och $\vec{v}$.
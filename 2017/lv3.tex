\documentclass{article}

\usepackage[T1]{fontenc}
\usepackage[utf8]{inputenc}
\usepackage{amsmath}
\usepackage{amssymb}
\usepackage{scalerel}

\title{SI LV3 Linjär Algebra}
\author{Gustav Örtenberg | \small{gusort@student.chalmers.se}}
\date{2017-11-14	}


\begin{document}
\maketitle
\section{}
Låt $\vec{u}$ och $\vec{v}$ vara två stycken vektorer.
\begin{itemize}
\item[a) ] Skriv upp definitionen för skalärprodukten mellan $\vec{u}$ och $\vec{v}$.
\item[b) ] Beräkna skalärprodukten mellan $\vec{u} = \begin{bmatrix} 1 \\ 4 \end{bmatrix}$ och $\vec{v} = \begin{bmatrix} 2 \\ 2 \end{bmatrix}.$
\item[c) ] Vad är resultatet av en skalärprodukt? Det vill säga, vad ger formeln er för någonting?
\item[d) ] Beräkna skalärprodukten av $\vec{v} \cdot \vec{v}$.
\item[e) ] Vad säger det er om skalärprodukten mellan $\vec{u}$ och $\vec{v}$ blir noll?
\end{itemize}


\section{}
Antag att ni befinner er i en luftballong påväg mot nordpolen. Under vindstilla förhållanden färdas ni med en hastighet av 30 km/h. Antag att det blåser en vind från väst som ger ballongen ett bidrag med en hastighet av 10 km/h.
\begin{itemize}
\item[a) ] Vad blir er fardt och hur mycket avviker er kurs rakt norrut?
\item[b) ] Antag att vinden istället är nordvästlig. Vad blir er fart och hur mycket avviker er kurs rakt norrut?
\item[c) ] Antag återigen att vinden är västlig. För att inte hamna ur kurs, krascha, och gå samma öde till mötes som Salomon August Andrées polarexpedition måste ni korrigera er kurs! Hur mycket måste ni ändra er riktning för att återigen flyga rakt norrut? Vad blir då er verkliga hastighet?
\end{itemize}


\section{}
Givet de två punkterna $(-3,4),\ (1,2)$ finn den linje som går genom bägge punkterna. Ligger punkten $(3,1)$ på linjen? Om inte ge en parrallell linje som går genom punkten istället. 

\section{}
Givet de tre punkterna $(1,2,3),\ (-2,6,1),\ (2,-9,3)$ finn de plan med samtliga punkter i. Ge ett exempel på en vektor som är ortogonal till planet.

\section{}
Två vektorer $\varphi,\ \chi$ är separerade med $\pi/6$ radianer. Givet $\varphi= \begin{bmatrix}3 \\ -2 \end{bmatrix},\ ||\chi|| = 8$ hur ser vektorn $\chi$ ut?

\section{}
Givet en bas i $\mathbb{R}^4,\ e_1=(2,3,1,6),\ e_2=(-3,2,-6,1),\ e_3=(-1,6,2,-3),\ e_3=(-6,-1,3,2)$, undersök om koordinataxlarna är ortogonala. Om så normera basen. Oavsett ortogonalitet för basen, kan man uttrycka vektorn $\begin{bmatrix}1\\2\\3\\6\end{bmatrix}$ i basen?

\section{}
{\it Tenta IT, 2015 augusti, uppg. 5}\\
Finn en positivt orienterad ON-bas $\vec{g_1},\vec{g_2}$ och $\vec{g_3}$, där $\vec{g_2}$ har samma riktning som $\begin{bmatrix}1\\-1\\0\end{bmatrix}$, och $\vec{g_3}$ har samma riktning som $\begin{bmatrix}2\\2\\1\end{bmatrix}$. Beräkna koordinaterna för vektorn $\begin{bmatrix}1\\2\\3\end{bmatrix}$ i basen $\vec{g_1},\vec{g_2}$ och $\vec{g_3}$.



\end{document}
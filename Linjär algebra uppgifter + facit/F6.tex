\begin{enumerate}
\item Ta produkten av de tal som är i samma plan och summera alla dessa produkter. Ex en 2d vektors skalärprodukt är $v \cdot u = v_{x}*u_{x}+v_{y}*u_{y} $ alternativt om man känner vinkeln mellan vektorerna $v \cdot u = ||v||\cdot||u||\cos\theta$
\item $2+8=10$. Alternativt ta fram vinkeln och längderna, använd formeln och få samma resultat. 
\item Poängen här är att skalärprodukten ger er ett tal, och inte en vektor. Den ger er även information om vinkeln mellan vektorerna. Man kan också se det som att det är den ena vektorn A's projektion på B gånger B's längd. 
\item $4+4=8$ Det vill säga längden av $\vec{v}$ i kvadrat.
\item De är ortogonala.
\end{enumerate}
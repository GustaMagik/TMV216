Vi har att $\cos\pi/4=1/\sqrt{2}$ så endligt definitionen av skalärprodukt så har vi\\
$\frac{1}{\sqrt{2}}=\frac{\vec{v}\cdot\vec{w}}{||\vec{v}||\ ||\vec{w}||}=\frac{\vec{v}\cdot\vec{w}}{3\ ||\vec{w}||}\iff\frac{3}{\sqrt{2}}=\frac{\vec{v}\cdot\vec{w}}{||\vec{w}||}$\\
Man kan ansätta en godtycklig vektor $\vec{w}=(w_1, w_2, w_3)$ och sätta in i ekvationen. Det finns oändligt många lösningar och man kan t ex välja att sätta $w_3=0$ och får då ekvationen\\
$w_1^2+w_2^2-16w_1w_2=0\iff(w_1-8w_2)^2=(\sqrt{63}w_2)^2\iff w_1=(8\pm\sqrt{63})w_2$\\
så en möjlig lösning är\\
$\vec{w}=\begin{bmatrix}8\pm\sqrt{63}\\1\\0\end{bmatrix}$\\
Ett annat alternativ är att först bestämma en vecktor $\vec{x}$ som är ortogonal mot $\vec{v}$ och har samma längd. En vektor med sökta egenskapen är då $\vec{w}=\vec{v}+\vec{x}$ (eftersom $\vec{w}$ blir diagonalen i kvadraten som spänns upp av $\vec{v}$ och $\vec{x}$). Man kan t ex ta\\
$\vec{x}=\frac{3}{\sqrt{2}}\begin{bmatrix}1\\-1\\0\end{bmatrix}$ så att $\vec{w}=\begin{bmatrix}2\\2\\1\end{bmatrix}+\frac{3}{\sqrt{2}}\begin{bmatrix}1\\-1\\0\end{bmatrix}=\begin{bmatrix}2+3/\sqrt{2}\\2-3/\sqrt{2}\\1\end{bmatrix}$.\\
Ytterligare en möjlighet är att bestämma matrisen A för en linjär avbildning som roterar kring en axel som är ortogonal mot $\vec{v}$. Svaret blir då t ex $\vec{w}=A\vec{v}$.

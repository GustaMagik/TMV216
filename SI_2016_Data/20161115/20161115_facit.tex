\documentclass{article}

\usepackage[T1]{fontenc}
\usepackage[utf8]{inputenc}
\usepackage[swedish]{babel}
\usepackage{amsmath}
\usepackage{amssymb}
\usepackage{scalerel}

\DeclareMathOperator*{\Bigcdot}{\scalerel*{\cdot}{\bigodot}}


\title{Supplemental Instructions}
\author{Niklas Gustafsson \\ 
		\small{niklgus@student.chalmers.se}
}
\date{
      2016-11-15
     }

\begin{document}
\maketitle
\subsection*{Vektorrummet}
\subsubsection*{1.}
\begin{itemize}
\item[a) ] $\begin{bmatrix}-4+6\\-2-12\\4+9\\6-12\\12+3\end{bmatrix}=\begin{bmatrix}2\\-14\\13\\-6\\15\end{bmatrix}$
\item[b) ] $\begin{bmatrix}-6-4\\-3+8\\6-6\\9+8\\18-2\end{bmatrix}=\begin{bmatrix}-10\\5\\0\\17\\16\end{bmatrix} $
\item[c) ] $\vec{v}\cdot\vec{u}=4-4-6+12-6=0$ Ja!
\item[d) ] $\mathbb{R}^5$
\end{itemize}


\subsection*{Linjära ekvationssystem}
\subsubsection*{2.}

\begin{itemize}
\item[a) ] 2.
\item[b) ] $\begin{bmatrix}\vec{v_x} & \vec{v_y}\end{bmatrix}$
\item[c) ] $\begin{bmatrix}1&-2&5\\4&1&3\end{bmatrix}\thicksim\begin{bmatrix}1&-2&5\\0&9&-17\end{bmatrix}\thicksim\begin{bmatrix}1&0&\frac{47}{9}\\0&1&\frac{-17}{9}\end{bmatrix}$
\item[d) ] Linjerna $x-2y=5,\ 4x+y=3$ skär i punkten $(\frac{47}{9},\frac{-17}{9} )$
\end{itemize}

\subsubsection*{3.}
$\begin{bmatrix}
1 &2 &-1 \\
-1 &3 & 3
\end{bmatrix}\thicksim
\begin{bmatrix}
1 & 2&-1 \\
0 &5 &2 
\end{bmatrix}\thicksim
\begin{bmatrix}
1 &0 &\frac{-9}{5} \\
0 &1 &\frac{2}{5} 
\end{bmatrix}
$ Linjerna korsar varandra i $(\frac{-9}{5},\frac{2}{5})$

\subsubsection*{4.}
$\begin{bmatrix}
4 &-8 &8 \\
-1 &2 &2 
\end{bmatrix}\thicksim
\begin{bmatrix}
0 &0 &16 \\
 1&-2 &-2 
\end{bmatrix}
$ Lösning saknas, lijnerna korsar ej varandra.

\subsubsection*{5.}
$\begin{bmatrix}
2&-3&1&8\\
-2&4&-1&2\\
4&-6&2&16
\end{bmatrix}\thicksim
\begin{bmatrix}
2&-3&1&8\\
0&1&0&10\\
0&0&0&0
\end{bmatrix}\thicksim
\begin{bmatrix}
2&0&1&38\\
0&1&0&10\\
0&0&0&0
\end{bmatrix}
$\\ Oändligt antal lösningar. Sätt $z=t$ och erhåll $x=19-t/2,\ y=10,\ z=t$

\subsubsection*{6.}
$\begin{bmatrix}
4 &-5 &1 &3 \\
-2&2 &-2 &11 \\
12 &-14 &3 &1
\end{bmatrix}\thicksim
\begin{bmatrix}
0 &-1 &-3 &25 \\
-2 &2 &-2 &11 \\
0 &-2 &-9 &67
\end{bmatrix}\thicksim
\begin{bmatrix}
0 &-1 &-3 &25 \\
-2 &0 &-8 &61 \\
0 &0 &-3 &17
\end{bmatrix}\thicksim
\begin{bmatrix}
0 &-1 &-3 &25 \\
1 &0 &4 &-\frac{61}{2} \\
0 &0 &1 &-\frac{17}{3}
\end{bmatrix}\thicksim
\begin{bmatrix}
0 &1 &0 &8 \\
1 &0 &0 &-\frac{47}{6}\\
0 &0 &1 &-\frac{17}{3}
\end{bmatrix}
$ De tre linjerna möts i punkten $(x=-\frac{47}{6},\ y=-8,\ z=-\frac{17}{3})$

\subsubsection*{7.}
$\begin{bmatrix}
2&3&-1&1\\
-1&4&-2&-2\\
3&-1&1&3
\end{bmatrix}\thicksim
\begin{bmatrix}
0&11&-5&-3\\
-1&4&-2&-2\\
0&11&-5&-3
\end{bmatrix}\thicksim
\begin{bmatrix}
0&11&-5&-3\\
-1&4&-2&-2\\
0&0&0&0
\end{bmatrix}\thicksim
\begin{bmatrix}
0&11&-5&-3\\
-1&0&-\frac{2}{11}&-\frac{10}{11}\\
0&0&0&0
\end{bmatrix}
$\\
Sätt $x_3=s,\ x_4=t$ och erhåll $\begin{bmatrix}x_1\\x_2\\x_3\\x_4\end{bmatrix}=s\cdot\begin{bmatrix}-\frac{2}{11}\\\frac{5}{11}\\1\\0\end{bmatrix}t\cdot\begin{bmatrix}-\frac{10}{11}\\\frac{3}{11}\\0\\1\end{bmatrix}$
\\\\
Geometrisk tolkning är ett plan i $\mathbb{R}^4$ som korsar origo.
\subsubsection*{8.}
Ställ upp som ekvationsystem och Gausseliminera. Värdena skall bli:\\
Kaffe: 7kr\\
Te: 5kr\\
Körsbärspaj: 8kr\\
Choklad: 12kr\\
Kanelbulle: 7kr

\subsubsection*{9.}
Bestäm normalekvationen: $A^T\cdot A\mathbf{x}+A^T\cdot\mathbf{b}$.\\
$A^T\cdot A\mathbf{x}=\begin{bmatrix}1&2&3&4&5\\1&1&1&1&1\end{bmatrix}\begin{bmatrix}1&1\\2&1\\3&1\\4&1\\5&1\end{bmatrix}=\begin{bmatrix}55&15\\15&5\end{bmatrix}$\\
$A^T\cdot\mathbf{b}=\begin{bmatrix}1&2&3&4&5\\1&1&1&1&1\end{bmatrix}\begin{bmatrix}2\\4\\5\\7\\8\end{bmatrix}=\begin{bmatrix}93\\26\end{bmatrix}$
$\begin{bmatrix}55&15&93\\15&5&26\end{bmatrix}\thicksim\begin{bmatrix}10&0&15\\15&5&26\end{bmatrix}\thicksim\begin{bmatrix}1&0&\frac{3}{2}\\0&5&\frac{7}{2}\end{bmatrix}$\\
Vilket ger $k=\frac{3}{2},\ m=\frac{7}{10}$.
\end{document}
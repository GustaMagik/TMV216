\documentclass{article}

\usepackage[T1]{fontenc}
\usepackage[utf8]{inputenc}
\usepackage[swedish]{babel}
\usepackage{amsmath}
\usepackage{scalerel}

\DeclareMathOperator*{\Bigcdot}{\scalerel*{\cdot}{\bigodot}}


\title{Supplemental Instructions}
\author{Niklas Gustafsson \\ 
		\small{niklgus@student.chalmers.se}
}
\date{
      2016-11-08
     }

\begin{document}
\maketitle
\subsubsection*{1.}
\begin{itemize}
\item[a) ] $\vec{v} \cdot \vec{u} = \sum \vec{v}_{i}\vec{u}_{i}$
\item[b) ] $\vec{v} \cdot \vec{u} = 2+8 = 10$
\item[c) ] Skalärproduketen är en vektors längd gånger en annan vektors ortogonalprojektion på denna.
\item[d) ] $\vec{v} \cdot \vec{v} = 4+4 = 8$.
\item[e) ] Vektorerna är ortogonala.
\end{itemize}

%Ortogonal projektion
\subsubsection*{2.}
\begin{itemize}
\item[a) ] $\alpha = \arccos{\frac{||\vec{u}|| \cdot ||\vec{v}||}{\vec{u} \cdot \vec{v}}} = \arccos{\frac{6}{12}}=\arccos{0.5}=60^{\circ}$ 
\item[b) ] $\vec{v_{L}}=\frac{\vec{u}\cdot\vec{v}}{\vec{u}\cdot\vec{u}}\vec{u}=\frac{6}{18}\begin{bmatrix}3 \\ 3 \end{bmatrix}= \begin{bmatrix}1 \\ 1 \end{bmatrix}$
\item[c) ] $\vec{w_{L}}=\frac{\vec{u}\cdot\vec{w}}{\vec{u}\cdot\vec{u}}\vec{u}=\frac{-12}{18}\begin{bmatrix}3 \\ 3 \end{bmatrix}= \begin{bmatrix}-2 \\ -2 \end{bmatrix}$
\item[d) ] $||\vec{v_{L}}||= \sqrt{2},\ ||\vec{w_{L}}||=2\sqrt{2} $
\item[e) ] $\vec{v_{L}}=\frac{\vec{u}\cdot\vec{v}}{\vec{u}\cdot\vec{u}}\vec{u}\ :\ \{ \vec{v}=\begin{bmatrix}1 \\ 0 \end{bmatrix},\ \vec{u}=\begin{bmatrix}0 \\ 1 \end{bmatrix}\} =\frac{0}{1}\begin{bmatrix}0 \\ 1 \end{bmatrix}= \begin{bmatrix}0 \\ 0 \end{bmatrix}$
\item[f) ] $\vec{v_{S}}=2\vec{v_{L}}-\vec{v}\ :\ \{ \vec{v}=\begin{bmatrix}1 \\ 0 \end{bmatrix},\ \vec{u}=\begin{bmatrix}0 \\ 1 \end{bmatrix} \} = 2\begin{bmatrix}0 \\ 0 \end{bmatrix}- \begin{bmatrix}1 \\ 0 \end{bmatrix}= \begin{bmatrix}-1 \\ 0 \end{bmatrix}= \vec{-v}$
\end{itemize}   

\subsubsection*{3.}
Parrallellförflyttning av $\vec{v} \to \begin{bmatrix}-1+0 \\ 3+5\end{bmatrix}=\begin{bmatrix}-1 \\ 8\end{bmatrix},\ L=\begin{bmatrix}1 \\ 4\end{bmatrix}t+\begin{bmatrix}0 \\ 5\end{bmatrix}$
\begin{itemize}
\item[a) ] Vi tar rikningsvektorn för L $\vec{u}=\begin{bmatrix}1 \\ 4\end{bmatrix}$ och projicerar $\vec{v}$ på denna. $\vec{v_{L}}=\frac{\vec{u}\cdot\vec{v}}{\vec{u}\cdot\vec{u}}\vec{u}=\frac{11}{17}\begin{bmatrix}1 \\ 4 \end{bmatrix}$
\item[b) ] $\vec{v_{S}}=2\vec{v_{L}}-\vec{v}= \frac{22}{17}\begin{bmatrix}1 \\ 4 \end{bmatrix}- \begin{bmatrix}-1 \\ 3 \end{bmatrix} =\frac{1}{17} \begin{bmatrix} 39 \\ 37\end{bmatrix}$
\end{itemize}

\subsubsection*{4.}
Vektorn mellan de två punkterna fås genom deras differens $\vec{v}=\begin{bmatrix} 4-1\\1-3 \end{bmatrix}= \begin{bmatrix} 3\\-2 \end{bmatrix}$ Detta blir linjens rikningsvektor $\vec{r}=\begin{bmatrix} 3\\-2 \end{bmatrix}$ till detta behövs den punkten vi subtraherade med för att kunna skriva linjen på parameterform som $L=\begin{bmatrix} 3t+1 \\-2t+3\end{bmatrix}$.
\\
På normarlform skrivs linjens ekvation $L=Ax+By=C$ som i vårat fall fås av normalen till $\vec{r}$ är $\vec{n}=\begin{bmatrix} -2\\-3 \end{bmatrix}$ (skalärprodukt=0). Finn sedan C så att en av punkterna finns på linjen $=-2x-3y=C$ tag punkten $\begin{bmatrix}1\\3\end{bmatrix}$ exempelvis $L=-2-9=-11=C$. Detta ger $L=-2x-3y=-11$ som normalform för linjen.

\subsubsection*{5.}
\begin{itemize}
\item[a) ] 
\item[b) ] 
\item[c) ] 
\item[d) ] 
\item[e) ] 
\item[f) ] 
\end{itemize}

\subsection*{6.}
\begin{itemize}
\item[a) ] $\vec{v}+\vec{u}=\begin{bmatrix}0 \\ 6 \\ 1 \\ 8 \\ 5 \end{bmatrix}$
\item[b) ] $\vec{v}-\vec{u}=\begin{bmatrix}2 \\ -2 \\ 5 \\ 0 \\ 5 \end{bmatrix}$
\item[c) ] $\alpha = \arccos{\frac{||\vec{u}|| \cdot ||\vec{v}||}{\vec{u} \cdot \vec{v}}} = \arccos{\frac{\sqrt{55}}{17}}=1.18^{\circ}$ 
\item[d) ] $\vec{v} \in \mathbf{R}^{5}$
\end{itemize}

\end{document}
Vektorn mellan de två punkterna fås genom deras differens $\vec{v}=\begin{bmatrix} 4-1\\1-3 \end{bmatrix}= \begin{bmatrix} 3\\-2 \end{bmatrix}$ Detta blir linjens rikningsvektor $\vec{r}=\begin{bmatrix} 3\\-2 \end{bmatrix}$ till detta behövs den punkten vi subtraherade med för att kunna skriva linjen på parameterform som $L=\begin{bmatrix} 3t+1 \\-2t+3\end{bmatrix}$.
\\
På normarlform skrivs linjens ekvation $L=Ax+By=C$ som i vårat fall fås av normalen till $\vec{r}$ är $\vec{n}=\begin{bmatrix} -2\\-3 \end{bmatrix}$ (skalärprodukt=0). Finn sedan C så att en av punkterna finns på linjen $=-2x-3y=C$ tag punkten $\begin{bmatrix}1\\3\end{bmatrix}$ exempelvis $L=-2-9=-11=C$. Detta ger $L=-2x-3y=-11$ som normalform för linjen.
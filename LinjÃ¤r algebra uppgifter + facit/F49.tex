Radoperationen (rad 3) - 3*(rad 1) ger oss det ekvivalenta systemet nedan.


$\begin{bmatrix}1&-2&\text{b}-2&-1\\0&\text{a+b}&-1&2\\0&0&6-2\text{b}&\text{a}+8\\ \end{bmatrix}$

Om b = 3 ser vi att lösningar saknas om inte a = -8, i vilket fall vi har oändligt många lösningar (en fri kolumn och därmed en linje som lösningsmängd). Om b!= 3, så har vi exakt en lösning om inte a+b = 0. För att undersöka detta fall sätter vi a = -b och får genom radoperationen rad 3 = (rad 2) + (6-2b) * (rad 1) systemet nedan.

$\begin{bmatrix}1&-2&\text{b}-2&-1\\0&0&-1&2\\0&0&0&20-5\text{b}\\ \end{bmatrix}$
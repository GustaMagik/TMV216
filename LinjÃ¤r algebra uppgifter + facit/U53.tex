Biluthyrningsfirman Hyr-Ett-Vrak har två kontor, ett på Centralen och ett på Landvetter. Av de bilar som är på Centralen i början av en vecka är $70\%$ kvar där i början av veckan därpå, $10\%$ finns på Landvetter och $20\%$ är uthyrda. FÖr Landvetter är motsvarande siffror att $60\%$ är kvar på Landvetter $10\%$ är på Centralen och $30\%$ är uthyrda. Av de som var uthyrda i början av en vecka är $50\%$ det också veckan därpå, $30\%$ är på Centralen och $20\%$ på Landvetter. Låt $c_n$ vara andelen bilar på Centralen vecka $n,\ l_n$ andelen bilar på Landvetter vecka $n$ och $u_n$ andelen uthyrda bilar vecka $n$ och låt\\
$$\vec{v_n}=\begin{bmatrix}c_u\\l_n\\u_n\end{bmatrix}$$\\
\begin{itemize}
	\item[a) ] Bestäm matris $A$ sådan att $\vec{v_n}=A\vec{v_{n.1}}$. Uttryck $\vec{v_n}$ i termer av $A$, $\vec{v_0}$ och $n$.
	\item[b) ] Beräkna en stationär fördelningav bilarna. (Notera att vi använder kolumnvektorer här, inte radvektorer som vid Markovkedjor.)
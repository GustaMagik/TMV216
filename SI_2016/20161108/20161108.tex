\documentclass{article}

\usepackage[T1]{fontenc}
\usepackage[utf8]{inputenc}
\usepackage[swedish]{babel}
\usepackage{amsmath}
\usepackage{scalerel}

\DeclareMathOperator*{\Bigcdot}{\scalerel*{\cdot}{\bigodot}}


\title{Supplemental Instructions}
\author{Niklas Gustafsson \\ 
		\small{niklgus@student.chalmers.se}
}
\date{
      2016-11-08
     }

\begin{document}
\maketitle
\subsection*{Skalärprodukt}
\subsubsection*{1.}
Låt $\vec{u}$ och $\vec{v}$ vara två stycken vektorer.
\begin{itemize}
\item[a) ] Skriv upp definitionen för skalärprodukten mellan $\vec{u}$ och $\vec{v}$.
\item[b) ] Beräkna skalärprodukten mellan $\vec{u} = \begin{bmatrix} 1 \\ 4 \end{bmatrix}$ och $\vec{v} = \begin{bmatrix} 2 \\ 2 \end{bmatrix}.$
\item[c) ] Vad är resultatet av en skalärprodukt? Det vill säga, vad ger formeln er för någonting?
\item[d) ] Beräkna skalärprodukten av $\vec{v} \cdot \vec{v}$.
\item[e) ] Vad säger det er om skalärprodukten mellan $\vec{u}$ och $\vec{v}$ blir noll?
\end{itemize}

\subsection*{Ortogonal projektion}
\subsubsection*{2.}
Låt L vara en linje i planet med riktningsvektor $\vec{u} = \begin{bmatrix} 3 \\ 3 \end{bmatrix}$. Låt $\vec{v} = \begin{bmatrix} 0 \\ 2 \end{bmatrix}$ och låt $\vec{w} = \begin{bmatrix} -4 \\ 0 \end{bmatrix}$ .
\begin{itemize}
\item[a) ] Vad är vinkeln $\alpha$ mellan $\vec{u}$ och $\vec{v}$?
\item[b) ] Vad är den ortogonala projektionen av $\vec{v}$ på L?
\item[c) ] Vad är den ortogonala projektionen av $\vec{w}$ på L?
\item[d) ] Vad blir längderna på dessa projektioner?
\item[e) ] Bevisa att den ortogonala projektionen av en vektor $\vec{v}$ på en linje med riktningsvektor $\vec{u}$ alltid är nollvektorn om de är ortogonala mot varandra. Kan ni även motivera detta grafiskt? 
\item[f) ] Bevisa att speglingen av en vektor $\vec{v}$ på en linje med riktningsvektor $\vec{u}$ alltid är $-\vec{v}$ om de är ortogonala mot varandra. Kan ni även motivera detta grafiskt?
\end{itemize}   

\subsection*{Linjer}
\subsubsection*{3.}
Antag att ni har en linje $y = 4 \cdot x + 5$ och en vektor $\vec{v} = \begin{bmatrix} -1 \\ 3 \end{bmatrix}$. Parallelförflytta denna vektor så att den har sin utgångspunkt i samma punkt som linjen korsar y-axeln.
\begin{itemize}
\item[a) ] Vad blir den ortogonala projektionen av vektorn på linjen efter förflyttningen? Tips: ta fram riktningsvektorn för linjen. 
\item[b) ] Vad blir speglingen av vektorn i linjen?
\end{itemize}

\subsubsection*{4.}
Antag att ni har punkterna $(1,3)$ och $(4,1)$. Ta fram ekvationen för den linje som går igenom desssa punkter. Skriv upp ekvationen både på normalform och på parameterform.  

\subsubsection*{5.}
Antag att ni har punkterna $P_0 = (1,2,3)$, $P_1 = (-1,-3,4)$ och $P_2 = (0,1,2)$.
\begin{itemize}
\item[a) ] Bestäm ekvationerna för de tre linjerna $L_1$, $L_2$ och $L_3$ som har riktningsvektorerna $\vec{P_0 P_1}$, $\vec{P_0 P_2}$ och $\vec{P_1 P_2}$. Skriv upp linjernas ekvationer både på normalform och parameterform. 
\item[b) ] Är några av linjerna parallella?
\item[c) ] Är några av linkerna vinkelräta mot varandra?
\item[d) ] Var korsar dessa linjer det "vanliga" xy-planet?
\item[e) ] Bestäm en riktningsvektor för en linje så att den blir ortogonal mot $L_1$. 
\item[f) ] Vad blir den ortogonala projektionen av denna riktgningsvektorn på $L_2$?
\end{itemize}

\subsection*{Vektorrummet $R^n$}
\subsubsection*{6.}
Antag att ni har vektorerna $\vec{v}=\begin{bmatrix}1 \\ 2 \\ 3 \\ 4 \\ 5 \end{bmatrix}$ och $\vec{u}=\begin{bmatrix}-1 \\ 4 \\ -2 \\ 4 \\ 0 \end{bmatrix}$.
\begin{itemize}
\item[a) ] Vad blir $\vec{v}+\vec{u}$?
\item[b) ] Vad blir $\vec{v}-\vec{u}$?
\item[c) ] Vad blir vinkeln mellan $\vec{v}$ och $\vec{u}$?
\item[d) ] Vad kallas det vektorrum som $\vec{v}$ tillhör? Hint: Hur många dimensioner har $\vec{v}$? 
\end{itemize}

\end{document}
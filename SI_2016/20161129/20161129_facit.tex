\documentclass{article}

\usepackage[T1]{fontenc}
\usepackage[utf8]{inputenc}
\usepackage[swedish]{babel}
\usepackage{amsmath}
\usepackage{scalerel}

\title{Supplemental Instructions}
\author{Niklas Gustafsson \\ 
		\small{niklgus@student.chalmers.se} \\
		Gustav Örtenberg \\ 
		\small{gusort@student.chalmers.se}
}
\date{
      2016-11-29
     }

\begin{document}
\maketitle

\subsection*{Determinanter}
Saurus regel.
\subsubsection*{1.}
En matris är inverterbar, omm dess determinant är nollskild.
\begin{itemize}
	\item[a) ] $5-6=-1$
	\item[b) ] $10-10=0$
	\item[c) ] $1*1*-1+2*9*3+3*2*6-(3*1*3+6*9*1+(-1)*2*2)=30$
	\item[d) ] $3*7*24+6*-2*4+9*2*11-(4*7*9+11*-2*3+24*2*6)=180$
	\item[e) ] $108 $
	\item[f) ] $0 $
\end{itemize}

\subsection*{Linjära avbildningar}
\subsubsection*{2.}
\begin{itemize}
	\item[a) ] Testa med 2 vektorer att funktionen $f()$ håller likheten $f(a\vec{x}+b\vec{y})=af(\vec{x})+bf(\vec{x})$ där a & b är skalärer.
	\item[b) ] $\begin{bmatrix}1+3\\1-3\end{bmatrix}=\begin{bmatrix}4\\-2\end{bmatrix}$
	\item[c) ] $A=\begin{bmatrix}1&1\\1&-1\end{bmatrix}$
	\item[d) ] Det stämmer!
\end{itemize}

\subsubsection*{3.}
\begin{itemize} 
	\item[a) ] Testa med 2 vektorer att funktionen $f()$ håller likheten $f(a\vec{x}+b\vec{y})=af(\vec{x})+bf(\vec{x})$ där a & b är skalärer.
	\item[b) ] 
	\item[c) ]  
	\item[d) ] 
	\item[e) ] 
	\item[f) ] 
\end{itemize}

\subsubsection*{4.}
Svaret ges av en sammansättning av matrisen A som roterar med $\phi=\frac{\pi}{3}$ och matrisen B som projicerar på y-axeln.\\ $A=\begin{bmatrix}\cos{\phi}&-\sin{\phi}\\\sin{\phi}&\cos{\phi}\end{bmatrix},\ B=\begin{bmatrix}0&0\\0&1\end{bmatrix}$ \\ Avbildningen ges av $BA=\begin{bmatrix}0&0\\\sin{\phi}&\cos{\phi}\end{bmatrix}$. Notera: Ordningen spelar ytterst stor roll.

\subsubsection*{5.}
\begin{itemize}
	\item[a) ] 
	\item[b) ] 
\end{itemize}

\subsection*{Affina avbildningar}
%En affin avbildning är en sammansättning av en linjär avbildning och en translation
\subsubsection*{6.}


\end{document}
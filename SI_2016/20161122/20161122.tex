\documentclass{article}

\usepackage[T1]{fontenc}
\usepackage[utf8]{inputenc}
\usepackage[swedish]{babel}
\usepackage{amsmath}
\usepackage{scalerel}

\DeclareMathOperator*{\Bigcdot}{\scalerel*{\cdot}{\bigodot}}


\title{Supplemental Instructions}
\author{Niklas Gustafsson \\ 
		\small{niklgus@student.chalmers.se} \\
		Gustav Örtenberg \\ 
		\small{gusort@student.chalmers.se}
}
\date{
      2016-11-22
     }

\begin{document}
\maketitle

\subsection*{Underrum, kolumnrum, rang och dimension}
%Nollrum och kolumnrum är så kallade underrum. Detta innebär att de är delmängder H till R^n (n är antal dimensioner)
%sådana att H innehåller nollvektorn, om u och v tillhör H så tillhör u+v H samt om u tillhör H tillör c*u också H dvs
%H innehåller alla linjärkombinationer av u och v. 
\subsubsection*{1.}
\begin{itemize}
	\item[a) ] Om ni tar fram nollrummet till en godtycklig matris A, vad får ni för något då?
	\item[b) ] Ta fram nollrummet till matrisen nedanför. \\
	$A=\begin{bmatrix}1 & 3 & 5\\ 2 & 4 & 6 \end{bmatrix}$
	\item[c) ] Om ni tar fram kolumnrummet till en godtycklig matris A, vad får ni för något då?
	\item[d) ] Ta fram kolumnrummet till matrisen ovan. 
	\item[e) ] Vad är rangen för A matrisen ovan? 
\end{itemize}

\subsection*{Matrisalgebra}
\subsubsection*{2.}
Låt A och B vara matriserna $A=\begin{bmatrix}1 & 4 & 7\\ 2 & 5 & 8 \\ 3 & 6 & 9\end{bmatrix}$ och $B=\begin{bmatrix}3 & -2 & -1\\ 2 & 1 & 9 \\ 3 & -3 & -1\end{bmatrix}$. Beräkna följande:

\begin{itemize}
	\item[a) ] $A+B$
	\item[b) ] $A-B$
	\item[c) ] $A \cdot B$
	\item[d) ] Vad är kravet för att en matris ska kunna kalls för symmetrisk? Är någon av matriserna A eller B symmetriska?
\end{itemize}

\subsection*{Determinanter}
\subsubsection*{3.}
\begin{itemize}
	\item[a) ] Beräkna determinanaten till matrisen $A = \begin{bmatrix} \vec{a_1} & \vec{a_2}\end{bmatrix} =\begin{bmatrix}1 & 2 \\ 3 & 5\end{bmatrix}$
	\item[b) ] Är $\vec{a_1}$ och $\vec{a_2}$ vänster eller högerorienterade?
	\item[c) ] Kluring: Antag att ni har matrisen $B = \begin{bmatrix} \vec{b_1} & \vec{b_2}\end{bmatrix} =\begin{bmatrix}1 & 2 \\ 1 & 2\end{bmatrix}$. Kan ni se vad determinanten av den här matrisen kommer att bli utan att beräkna den för hand?
	%Hint: Aren på parallellogrammet.
	\item[d) ] Beräkna $\frac{1}{det(A)} \cdot \begin{bmatrix}5 & -2 \\ -3 & 1\end{bmatrix}$. Multiplicera sedan resultatet med A. Märker ni någonting speciellt med det här resultatet?
\end{itemize}

\subsubsection*{4.}
\begin{itemize}
	\item[a) ] Beräkna determinanaten till matrisen $A=\begin{bmatrix}1 & 5 & 7\\ 2 & 9 & 8 \\ 3 & 3 & 9\end{bmatrix}$
\end{itemize}



\subsection*{Area, volym, kryssprodukt}
\subsubsection*{5.}
Antag att ni har matriserna $B=\begin{bmatrix} \vec{b_1} & \vec{b_2} & \vec{b_3} \end{bmatrix} = \begin{bmatrix}3 & -2 & -1\\ 2 & 1 & 9 \\ 3 & -3 & -1\end{bmatrix}$ och $A = \begin{bmatrix} \vec{a_1} & \vec{a_2}\end{bmatrix} =\begin{bmatrix} -3 & 2 \\ 7 & 5\end{bmatrix}$
\begin{itemize}
	\item[a) ] Beräkna arean av det parallellogram som spänns upp av $\vec{a_1}$ och $\vec{a_2}$.
	\item[b) ] Beräkna volymen av den parallellpiped som $\vec{b_1}$, $\vec{b_2}$ och $\vec{b_3}$ spänner upp.
	\item[c) ] En enhetskvadrat är en kvadrat vars sidor har längden 1. I det kartesiska planet har enhetskvadraten sina hörn i (0,0), (1,0), (0,1) och (1,1). Kan ni med hjälp av determinanter bevisa att arean av enhetskvadraten är 1? 
\end{itemize}

\subsubsection*{6.}
\begin{itemize}
	\item[a) ] Givet två vektorer $\vec{u}$ och $\vec{v}$, vad blir resultatet av kryssprodukten $\vec{u}$ x $\vec{v}$?
	\item[b) ] Beräkna kryssprodukten av $\vec{u} = \begin{bmatrix} 1 \\ 3 \\ 4\end{bmatrix}$ och $\vec{v} = \begin{bmatrix} 2 \\ 5 \\ 6\end{bmatrix}$
	\item[c) ] Beräkna kryssprodukten av $\vec{v} = \begin{bmatrix} 3 \\ 5 \\ -1 \end{bmatrix}$ och $\vec{u} = \begin{bmatrix} -2 \\ 7 \\ 1 \end{bmatrix}$. 
	\item[d) ] Beräkna kryssprodukten av $e_x = \begin{bmatrix} 1 \\ 0 \\ 0 \end{bmatrix}$, $e_y = \begin{bmatrix} 0 \\ 1 \\ 0 \end{bmatrix}.\\ \\ $ Kan ni gissa vad resultatet kommer att bli på förhand?
	\item[e) ] Kan ni representera följande vektorer i $R^3$ och beräkna kryss produkten av dem? $\vec{u} = \begin{bmatrix} 1 \\ 3 \end{bmatrix}$, $\vec{v} = \begin{bmatrix} 2 \\ 5 \end{bmatrix}$
\end{itemize}

\subsection*{Linjära ekvationssystem}
Repetition är all kunskaps moder!
\subsubsection*{7.}
Lös ekvationssystemet: 
\begin{math}
	\begin{cases}
	3 \cdot x - 1 \cdot y  + z = 8 \\
	-3 \cdot x + 2 \cdot y -z = 2 \\
	1 \cdot x - 5 \cdot y  + 2 \cdot z = 16 \\
	\end{cases}
	\\
	\\
\end{math}

\end{document}
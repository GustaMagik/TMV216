\documentclass{article}

\usepackage[T1]{fontenc}
\usepackage[utf8]{inputenc}
\usepackage[swedish]{babel}
\usepackage{amsmath}
\usepackage{scalerel}

\DeclareMathOperator*{\Bigcdot}{\scalerel*{\cdot}{\bigodot}}


\title{Supplemental Instructions Facit}
\date{
      2016-11-22
     }

\begin{document}
\maketitle

\section*{Underrum, kolumnrum, rang och dimension}
%Nollrum och kolumnrum är så kallade underrum. Detta innebär att de är delmängder H till R^n (n är antal dimensioner)
%sådana att H innehåller nollvektorn, om u och v tillhör H så tillhör u+v H samt om u tillhör H tillör c*u också H dvs
%H innehåller alla linjärkombinationer av u och v. 
\subsection*{1.}
\begin{itemize}
	\item[a) ] En linje som går igenom origo.
	\item[b) ] Gauss $~ \begin{bmatrix} 1&0 &-1 \\ 0& 1& 2 \end{bmatrix} => \begin{bmatrix}t \\-2t \\ t\end{bmatrix}$
	\item[c) ] De kolumnerna i A som vid fullständig gaussning är pivot kolumner.
	\item[d) ] $\begin{bmatrix}1 \\2 \end{bmatrix},\ \begin{bmatrix} 3\\ 4\end{bmatrix}$
	\item[e) ] 2, ty det finns 2st pivotkolumner.
\end{itemize}

\section*{Matrisalgebra}
\subsection*{2.}

\begin{itemize}
	\item[a) ] 	$\begin{bmatrix} 
				4&2 &6 \\
				4&6 &17\\
				6&3 &8
				\end{bmatrix}$
	\item[b) ] $\begin{bmatrix} 
				-2&6 &8 \\
				0&4 &-1\\
				0&9 &10
				\end{bmatrix}$
	\item[c) ] $\begin{bmatrix} 
				3+8+21 &-2+4-7 &-1+36-7\\
				6+10+24 &-4+5-24 &-3-18-9\\
				9+8+21 &-6+6-27 &-3+54-9
				\end{bmatrix}
				=\begin{bmatrix} 
				32 &-5 &28\\
				40 &-23 &-30\\
				38 &-27 &42
				\end{bmatrix}$
	\item[d) ] Nej, deras transponat är ej likt orginalmatriserna.
\end{itemize}

\section*{Determinanter}
\subsection*{3.}
\begin{itemize}
	\item[a) ] $5-6=-1$
	\item[b) ] $\vec{a_1}:\ vänsterorienterad,\ \vec{a_2}:\ högerorienterad$ ty determinaten är negativ.
	\item[c) ] 0, ty arean på parrallellogrammet är 0 då de 2 vektorerna ej spänner upp en area mellan varandra (de är parallella).
	\item[d) ] Det blir identitetsmatrisen.
\end{itemize}

\subsection*{4.}
\begin{itemize}
	\item[a) ]  $-60$
\end{itemize}

\section*{Area, volym, kryssprodukt}
\subsection*{5.}
\begin{itemize}
	\item[a) ] $Area = |det(A)| = 29$
	\item[b) ] $Volym = |det(B)| = 29$
	\item[c) ] Kan beskrivas som 2 vektorer $\vec{v_1}=\begin{bmatrix}1\\0\end{bmatrix},\ \vec{v_2}=\begin{bmatrix}0\\1\end{bmatrix}$ de tillsammans bildar en matris C och $Area = |det(C)| = 1$
\end{itemize}

\subsection*{6.}
\begin{itemize}
	\item[a) ] En ortogonal vektor vars längd är lika stor som arean som de två vektorerna spänner upp.
	\item[b) ] $\vec{u} \times \vec{v}=\begin{bmatrix}18-20\\8-6\\5-6\end{bmatrix}=\begin{bmatrix}-2\\2\\-1\end{bmatrix}$
	\item[c) ] $\vec{u}\times \vec{v}=\begin{bmatrix}5+7\\2-3\\21+10\end{bmatrix}=\begin{bmatrix}12\\-1\\31\end{bmatrix}$
	\item[d) ] $e_z=\begin{bmatrix}0\\0\\1\end{bmatrix}$, ja resultatet är en vektor ortogonal till de två man tar kryssprodukten av.
	\item[e) ] $\vec{u}=\begin{bmatrix}1\\3\\0\end{bmatrix},\ \vec{v}=\begin{bmatrix}2\\5\\0\end{bmatrix},\ 
	\vec{u}\times \vec{v}=\begin{bmatrix}0\\0\\5-6\end{bmatrix}=\begin{bmatrix}0\\0\\-1\end{bmatrix}$
\end{itemize}

\section*{Linjära ekvationssystem}
\subsection*{7.}
$\begin{bmatrix}
3 &-1 &1 & 8 \\
-3 &2 &-1 &2 \\
1 &-5 &2 & 16
\end{bmatrix}~
\begin{bmatrix}
0 &0 &1 &36\\
0 &1 &0 &10\\
1 &0 &0 &-6
\end{bmatrix}$

\end{document}
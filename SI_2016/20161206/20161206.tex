\documentclass{article}

\usepackage[T1]{fontenc}
\usepackage[utf8]{inputenc}
\usepackage[swedish]{babel}
\usepackage{amsmath}
\usepackage{scalerel}

\title{Supplemental Instructions}
\author{Niklas Gustafsson \\ 
		\small{niklgus@student.chalmers.se} \\
		Gustav Örtenberg \\ 
		\small{gusort@student.chalmers.se}
}
\date{
      2016-12-06
     }

\begin{document}
\maketitle

\subsection*{Linjärt beroende}
\subsubsection*{1.}
Undersök om vektorerna i respektive deluppgift är linjärt beroende eller linjärt oberoende.
%Lös ekvationen x1*v1+x2*v2+....+xn*vn = 0
%Om den enda lösningen är x1 = x2 = ... = xn = 0 är de linjärt oberoende
\begin{itemize}
	\item[a) ] $\vec{u} = \begin{bmatrix} 1 \\ 2 \\ 3 \end{bmatrix}$, $\vec{v} = \begin{bmatrix} 3 \\ 1 \\ 5 \end{bmatrix}$, $\vec{w} = \begin{bmatrix} 5 \\ 7 \\ 1 \end{bmatrix}$
	\item[b) ] $\vec{u} = \begin{bmatrix} 1 \\ 2 \\ 3 \end{bmatrix}$, $\vec{v} = \begin{bmatrix} 7 \\ 1 \\ 2 \end{bmatrix}$, $\vec{w} = \begin{bmatrix} 14 \\ 2 \\ 4 \end{bmatrix}$
	\item[c) ] $\vec{u} = \begin{bmatrix} 1 \\ 2 \\ 3 \end{bmatrix}$, $\vec{v} = \begin{bmatrix} 3 \\ 1 \\ 5 \end{bmatrix}$, $\vec{w} = \begin{bmatrix} 4 \\ 3 \\ 8 \end{bmatrix}$
	\item[d) ] $\vec{e_x}$, $\vec{e_y}$, $\vec{e_z}$
	\item[e) ] Kan ni kort och enkelt beskriva vad det innebär att två vektorer är linjärt beroende respektive oberoende?
	%Om de är linjärt beroende så kan de skrivas som linjärkombinationer av varandra. Om de är linjärt oberoende kan de inte det.
\end{itemize}

\subsection*{Baser och koordinater}
%Vektorerna utgör en bas om de är linjärt oberoende och varje vektor som tillhör mängden kan skrivas som en linjärkombination av dem
\subsubsection*{2.}
\begin{itemize}
	\item[a) ] Utifrån definitionen av en bas, vad är det som krävs för att vektorerna $\vec{v_1}, \vec{v_2} ... \vec{v_n}$ ska utgöra en bas i $R^n$? Uppfyller något av vektorparen i förra uppgiften dessa krav?
	\item[b) ] Ange en alternativ bas för $R^2$ (dvs inte $\vec{e_x}$ eller $\vec{e_y}$).
\end{itemize}

\subsubsection*{3.}
\textit{Uppgift ifrån tentamen 2016-01-04, gav två poäng.}
Vilken av följande vektoruppsättningar utgör \textbf{inte} en bas för $R^3$.
\begin{itemize}
	\item[a) ] $ \begin{bmatrix} 1 \\ 0 \\ 0 \end{bmatrix}$, $\begin{bmatrix} 0 \\ 1 \\ 0 \end{bmatrix}$, $\begin{bmatrix} 0 \\ 0 \\ 0 \\ 1 \end{bmatrix}$
	\item[b) ] $ \begin{bmatrix} 1 \\ 0 \\ 1 \end{bmatrix}$, $\begin{bmatrix} 2 \\ 0 \\ 2 \end{bmatrix}$, $\begin{bmatrix} 3 \\ 1 \\ 3 \end{bmatrix}$
	\item[c) ] $ \begin{bmatrix} 1 \\ 2 \\ 1 \end{bmatrix}$, $\begin{bmatrix} 1 \\ 1 \\ 2 \end{bmatrix}$
	\item[d) ] $ \begin{bmatrix} 1 \\ 0 \\ 1 \end{bmatrix}$, $\begin{bmatrix} 0 \\ 1 \\ 1 \end{bmatrix}$, $\begin{bmatrix} 1 \\ 1 \\ 0 \end{bmatrix}$, $\begin{bmatrix} 1 \\ 1 \\ 1 \end{bmatrix}$
	\item[e) ] $ \begin{bmatrix} 1 \\ 1 \\ 1 \end{bmatrix}$, $\begin{bmatrix} 2 \\ 1 \\ 1 \end{bmatrix}$, $\begin{bmatrix} 0 \\ 1 \\ 1 \end{bmatrix}$
\end{itemize}

\subsubsection*{4.}
\textit{Uppgift ifrån tentamen 2016-01-04, gav tre poäng} $\\ \\ $
Bestäm koordinaterna för vektorn $\vec{v} = \begin{bmatrix} 1 \\ 2 \\ 3 \end{bmatrix}$ relativt basen $F = \begin{bmatrix} 1 \\ 1 \\ 0 \end{bmatrix} \begin{bmatrix} 3 \\ 1 \\ 2 \end{bmatrix} \begin{bmatrix} 4 \\ -5 \\ 1 \end{bmatrix}$.

\subsubsection*{5.}
Låt G och F utgöra varsin bas i $R^3$ samt låt $\vec{v_F} = \begin{bmatrix} 1 \\ 2 \\ -1 \end{bmatrix}$. Bestäm $\vec{v_G}$. 

\noindent
$G = \begin{bmatrix} 1 \\ 1 \\ 0 \end{bmatrix} \begin{bmatrix} 3 \\ 1 \\ 2 \end{bmatrix} \begin{bmatrix} 4 \\ -5 \\ 1 \end{bmatrix}$ $F = \begin{bmatrix} 2 \\ 0 \\ 0 \end{bmatrix} \begin{bmatrix} 1 \\ 1 \\ 1 \end{bmatrix} \begin{bmatrix} 0 \\ 0 \\ -3 \end{bmatrix}$ 

\newpage

\subsection*{Egenvärden och egenvektorer}
\subsubsection*{6.}
Bestäm egenvärden och egenvektorer till matrisen $\begin{bmatrix} 4 & 5 \\ 5 & 4 \end{bmatrix}$.

\subsubsection*{7.}
\textit{Uppgift ifrån tentamen 2016-04-07, gav tre poäng.} \\ \\ 
Bestäm egenvärden och egenvektorer till produkterna $A \cdot B$ och $B \cdot A$. \\ \\ 
$A = \begin{bmatrix} 1 & 0 \\ 1 & 1 \end{bmatrix} B = \begin{bmatrix} 3 & -2 \\ -2 & 2 \end{bmatrix}$


\end{document}
